\documentclass[11pt]{article}
% General document formatting
\usepackage[margin=0.75in]{geometry}
\usepackage[parfill]{parskip}
\usepackage[utf8]{inputenc}
\usepackage{subfig}         % side-by-side figures 
% Related to math
\usepackage{amsmath,amssymb,amsfonts,amsthm}
\usepackage{graphicx}
\usepackage{natbib}
\usepackage{titling}
\usepackage{hyperref}
\usepackage{wrapfig}
\usepackage{algpseudocode}

\usepackage{booktabs} % for wrapping tabulars in accord with
\bibliographystyle{agu}
\setlength{\droptitle}{-5em}   % This is your set screw

%\usepackage[math]{kurier}
\newcommand\be{\begin{equation}} % shortcut to start eq envs 
\newcommand\ee{\end{equation}}   % shortcut to end eq envs
\newcommand\ol{\overline}        % shortcut to draw overline 
\newcommand\bra{\langle}
\newcommand\ket{\rangle}
\newcommand\El{\mathcal{L}}
\newcommand\tg{\tilde{g}}
\newcommand\tG{\tilde{G}}
\date{} % make the date field blank




% for pseudo code -- example from https://tex.stackexchange.com/questions/163768/write-pseudo-code-in-latex
\makeatletter
\def\BState{\State\hskip-\ALG@thistlm}
\makeatother

\begin{document}
\title{Auxiliary Material for "Joint stochastic model of bedload transport and bed elevations: derivation of heavy-tailed resting times" by James K. Pierce and Marwan A. Hassan}
\maketitle

\section*{Detailed description of Gillespie simulation algorithm}

The Gillespie Stochastic Simulation Algorithm (SSA) generates exact realizations of a Markov random process from a sequence of random numbers.
It was originally developed for chemical physics by \citet{Gillespie1977} and is reviewed in \citet{Gillespie1992} and \citet{Gillespie2007}.
The SSA hinges on the defining property of a Markov process. When the transition rates from one state to another are not dependent on the distant past, the process is Markov \citep[e.g.,][]{Cox1965}.
In the following sections, we begin by demonstrating the time intervals $\tau$ between subsequent transitions (i.e., the intervals between transition times) are exponentially distributed within the model we develop in the main text. Then we describe the SSA as a consequence of this property.

\subsection*{Times between transitions of any kind}
Our joint stochastic description of bedload and bed elevation changes is characterized by a set of states $(n,m)$ where $n$ and $m$ are positive integers. 
Our description involves four possible transitions (migration in, entrainment, deposition, migration out) with rates given in equations (2-5) in the main text.
From the state $(n,m)$, the rate (probability per unit time) for any transition to occur is the sum over all possibilities:
\begin{multline} A(n,m) = R_{MI}(n+1,m|n,m) + R_E(n+1,m-1|n,m) \\+ R_D(n-1,m+1|n,m) + R_{MO}(n-1,m|n,m).\end{multline}
Using this, the probability that no transition occurs from the state $(n,m)$ in a small time interval $\delta \tau$ is $1-A(n,m)\delta \tau$. If we denote by $Q(\tau|n,m)$ the probability density that a transition of any kind occurs from the state $(n,m)$ after a time $\tau$, we can express the probability density that a transition happens after a slightly larger time $\tau + \delta \tau$ as 
\be Q(\tau+\delta \tau|n,m) = \big[1-A(n,m)\delta \tau\big]Q(\tau|n,m).\ee
This equation invokes the Markov property, since it does not involve the past history of states $(n,m)$. Taking $\delta\tau \rightarrow 0 $ we find the master equation $\frac{d}{d\tau}Q(\tau|n,m) = -A(n,m)Q(\tau|n,m)$, from which we conclude the time $\tau$ between subsequent transitions is distributed as 
\be Q(\tau|n,m) = A(n,m)e^{-A(n,m)\tau}. \label{eq:exp}\ee
Therefore we have shown the time $\tau$ to the next transition from a state $(n,m)$ is exponentially distributed with mean value $\bar{\tau} = 1/A(n,m).$ In deriving this result, we used the normalization condition $1 = \int_0^\infty Q(\tau|n,m)d\tau.$

Crucially, this result implies if the stochastic process is in the state $(n,m)$ at a time $t$, the next transition will occur at a time $t+\tau$ with $\tau$ a random variable drawn from the exponential distribution \ref{eq:exp}.
Therefore we can determine the times between subsequent transitions by drawing exponentially distributed random numbers.

\subsection*{Selection of transitions that occur}

So far, we have determined how to step from one transition to the next, but we have not specified the type of transitions that occur.
Intuitively, this will depend on the relative magnitudes of the rates from equations (2-5) in the main text: the transition with the highest rate is most likely to occur. This is formalized by generating the ratios
\be S = \Bigg\{\frac{R_{MI}(n+1,m|n,m)}{ A(n,m)},\frac{R_E(n+1,m-1|n,m)}{ A(n,m)},\frac{R_D(n-1,m+1|n,m)}{ A(n,m)},\frac{R_{MO}(n-1,m|n,m)}{ A(n,m)}\Bigg\}. \label{eq:rel}\ee
By construction, $\text{sum}\{S\}=1$.
By forming cumulative sums of the four ratios, we partition the unit interval $[0,1]$ into four chunks, each associated with a transition -- either migration in, entrainment, deposition, or migration out. The transitions with the highest rates have the largest associated chunks. To randomly select the transition that occurs at a transition time, we draw a random number on $[0,1]$ and find which chunk it falls in.

In summary, to step the process through a single transition, we draw the time interval to the next transition from the distribution (\ref{eq:exp}), then draw a uniform random from $[0,1]$ and use it to select the transition that occurs from the cumulative sum of the ratios in (\ref{eq:rel}). The SSA simply iterates this random number generation/selection process to generate exact realizations of the stochastic process.

\subsection*{Pseudo code for the Gillespie SSA}

\begin{algorithmic}
	\State $t = 0$ 
	\State $n = n_0$\Comment{Set the initial state $(n_0,m_0)$}
	\State $m =  m_0$

	\While{$t<t_{\text{max}}$;}{	\Comment{Simulation will go until $t$ surpasses $t_\text{max}$}
	\State draw $\tau$ from eq. (\ref{eq:exp})
	\State $t = t+\tau$
	\State draw a random number $r$ in $[0,1]$
	\State compute the ratios $r_1,r_2,r_3,r_4$ in eq. (\ref{eq:rel})
	\State form the cumulative sums $r_i = \sum_{1\leq j\leq i}r_j$
	
	\If {$0<r<r_1$}	\Comment{Migration in}
	\State $n=n+1$
	\ElsIf {$r_1\leq r < r_2$} 	\Comment{Entrainment}
	\State $n= n + 1$
	\State $m = m-1 $
	\ElsIf {$r_2\leq r < r_3$} 	\Comment{Deposition}
	\State $ n = n - 1$
	\State $ m = m + 1$
	\ElsIf {$r_3 \leq r < 1$} \Comment{ Migration out}
	\State $ n = n-1$
	\EndIf
}
	\State record $(n,m,t)$ \Comment{Build time series of $n$ and $m$}
	\EndWhile
\end{algorithmic}

\bibliography{biblio}
\end{document}



